\documentclass{article}
\usepackage[utf8]{inputenc}
\usepackage[UTF8]{ctex}
\usepackage{geometry}
\usepackage{graphics}
\usepackage{float}
\usepackage{adjustbox}
\usepackage{fancyhdr}
\usepackage{titletoc}
\usepackage{amsmath}
\usepackage[labelfont=bf]{caption}


\title{\bfseries \Huge 力学讲义}  %题目
\author{}  %作者
\date{}  %日期
\fancyhf{}  
\fancyhead[C]{力学讲义}  %页眉
\fancyfoot[C]{\thepage}  %页脚(设定页码)
\pagestyle{fancy}

%设定目录格式
\titlecontents{chapter}
  [1em]
  {\bfseries\huge}
  {\contentslabel{1em}} 
  {\hspace*{0em}}
  {\titlerule*[0.9pc]{.}\contentspage}

\titlecontents{section}
  [2.5em] 
  {\bfseries\large}
  {\contentslabel{2.5em}} 
  {\hspace*{0em}}
  {\titlerule*[0.45pc]{.}\dotfill\contentspage}

\titlecontents{subsection}
  [4.1em] 
  {}
  {\contentslabel{3.5em}} 
  {\hspace*{0em}}
  {\titlerule*[0.44pc]{.}\dotfill\contentspage}

\begin{document}
  
\maketitle
\tableofcontents
\newpage

\section{数学知识}
    力学部分所需的数学知识并不复杂,一般而言,只要是在大学学过一学期的高等数学的课程,就可以
    轻松应对本部分出现的数学推导。对于高中生而言,也可以通过简单地接触一些高等数学的概念来快速上手。
    由于本讲义主要是以物理学概念的讲述为主,并不会很系统的涉及到数学内容,所以需要读者对以下知识有所了解。

    1.极限相关知识:包括简单的极限计算、泰勒展开公式等。

    2.微分、积分运算:包括熟悉导数和微分、能利用牛顿-莱布尼茨公式计算积分等。

    3.常微分方程:能用分离变量的方法解决简单的常微分方程。

    4.以及对未知事物的好奇心!

\section{质点动力学}
    质点是我们很熟悉的一个概念,高中物理中的绝大部分力学问题都是解决一个质点的运动。在本部分的内容中,
    我们将在高中学习内容的基础上,构建一个更加完善的理论体系,并补充一些物理的分析方法。

\subsection{运动的基本问题}
    在坐标系中描述一个质点的运动,我们常用的物理量是时间t、位置矢量\(\vec{r}\)、位移\(\vec{x}\)、速度\(\vec{v}\)和加速度\(\vec{a}\)。
    这些物理量之间的关系如下:
    \begin{align*}
        \vec{x} = \varDelta \vec{r}\ \ \  \vec{v}  = &\frac{d\vec{x}}{dt}\ \ \ \vec{a} = \frac{d\vec{v}}{dt} \\
        \vec{x} = x_0 + \int \vec{v} \,dt & \ \ \ \vec{v} = v_0 + \int \vec{a} \,dt 
    \end{align*}
    通常对于一个简单的直线运动,我们可以很容易地利用上述公式解决质点的运动问题。

    \textbf{例2.1.1}:一个质点在x轴上沿着正方向运动,其运动速度与时间的关系为\(v(t)=t+\sin t\),那么从t=0到\(t=\pi\)的时间内,
    质点运动的距离是多少?
    \begin{align*}
        \varDelta x = \int_0^{\pi} t+\sin t \,dt \\
        \varDelta x = \frac{1}{2}\pi^2 + 2
    \end{align*}

    进一步,我们可以来研究圆周运动。圆周运动的基本量是角度\(\theta\)、角速度\(\omega\)和角加速度\(\alpha\),
    这些物理量可以与上述的三个物理量相对应,其数学关系也是一致的,这里不再赘述。

\subsection{相对运动与参考系}
    一般我们在研究运动问题时,都是描述物体相对于大地的运动,这种方式可以很清楚地描述每个物体的具体运动方式。
    有时候我们可以用相对运动的方式去分析,这可以让我们把注意力放在需要被研究
    的对象上,而不用去考虑其他物体的运动,使得问题更加简洁。

    相对运动的理论基础是{\bfseries 伽利略变换},实际上,当我们说物体A相对于物体B的运动时,就是将物体A的参考系从大地参考系S,
    变换为了物体B的参考系\(S^\prime\)。

    \textbf{例2.2.1}:一个生活中常见的问题,在雨天中,经过相同的距离,是跑着淋雨更多,还是走着淋雨更多?(假设没有风,人可以抽象为立方体)

    人(作为一个立方体)在雨中前进,其运动方向前的那个面和上面是被雨淋的面。现在我们假设雨是静止的,人在充满雨滴的空间中运动,
    其运动速度是向前的速度\(v_x\)和向上的速度\(v_y\)。两种情况下运动距离x一致,但水平速度\(v_x\)不同,因而运动的时间t不同。




\section{刚体与刚体动力学}
    在高中阶段,大家研究的最多的物理模型是质点。质点是一个没有大小和形状的点,


\end{document}
