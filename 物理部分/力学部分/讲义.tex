\documentclass{article}
\usepackage[utf8]{inputenc}
\usepackage[UTF8]{ctex}
\usepackage{geometry}
\usepackage{graphics}
\usepackage{float}
\usepackage{adjustbox}
\usepackage{fancyhdr}
\usepackage{titletoc}
\usepackage{amsmath}
\usepackage{enumitem}
\usepackage[colorlinks=false]{hyperref}
\usepackage[labelfont=bf]{caption}


\title{\bfseries \Huge 力学讲义}  %题目
\author{bilibili:德布罗意不是波}  %作者
\date{}  %日期
\fancyhf{}  
\fancyhead[C]{力学讲义}  %页眉
\fancyfoot[C]{\thepage}  %页脚(设定页码)
\pagestyle{fancy}

%设定目录格式
\titlecontents{chapter}
  [1em]
  {\bfseries\huge}
  {\contentslabel{1em}} 
  {\hspace*{0em}}
  {\titlerule*[0.9pc]{.}\contentspage}

\titlecontents{section}
  [2.5em] 
  {\bfseries\large}
  {\contentslabel{2.5em}} 
  {\hspace*{0em}}
  {\titlerule*[0.45pc]{.}\dotfill\contentspage}

\titlecontents{subsection}
  [4.1em] 
  {}
  {\contentslabel{3.5em}} 
  {\hspace*{0em}}
  {\titlerule*[0.44pc]{.}\dotfill\contentspage}

\hypersetup{
    hidelinks,         
}

\begin{document}
  
\maketitle
\tableofcontents
\newpage

\section{数学知识}
    力学部分所需的数学知识并不复杂,一般而言,只要是在大学学过一学期的高等数学的课程,就可以
    轻松应对本部分出现的数学推导。对于高中生而言,也可以通过简单地接触一些高等数学的概念来快速上手。
    由于本讲义主要是以物理学概念的讲述为主,并不会很系统的涉及到数学内容,所以需要读者对以下知识有所了解。

    1.极限相关知识:包括简单的极限计算、泰勒展开公式等。

    2.微分、积分运算:包括熟悉导数和微分、能利用牛顿-莱布尼茨公式计算积分等。

    3.常微分方程:能用分离变量的方法解决简单的常微分方程。

    4.以及对未知事物的好奇心!

\section{质点动理学}
    质点是我们很熟悉的一个概念,高中物理中的绝大部分力学问题都是解决一个质点的运动。在本部分的内容中,
    我们将在高中学习内容的基础上,构建一个更加完善的理论体系,并补充一些物理的分析方法。

\subsection{运动的基本问题}
    在坐标系中描述一个质点的运动,我们常用的物理量是时间t、位置矢量\(\vec{r}\)、位移\(\vec{x}\)、速度\(\vec{v}\)和加速度\(\vec{a}\)。
    这些物理量之间的关系如下:
    \begin{align*}
        \vec{x} = \varDelta \vec{r}\ \ \  \vec{v}  = &\frac{d\vec{x}}{dt}\ \ \ \vec{a} = \frac{d\vec{v}}{dt} \\
        \vec{x} = x_0 + \int \vec{v} \,dt & \ \ \ \vec{v} = v_0 + \int \vec{a} \,dt 
    \end{align*}
    通常对于一个简单的直线运动,我们可以很容易地利用上述公式解决质点的运动问题。

    \textbf{例2.1.1}:一个质点在x轴上沿着正方向运动,其运动速度与时间的关系为\(v(t)=t+\sin t\),那么从t=0到\(t=\pi\)的时间内,
    质点运动的距离是多少?
    \begin{align*}
        \varDelta x = \int_0^{\pi} t+\sin t \,dt \\
        \varDelta x = \frac{1}{2}\pi^2 + 2
    \end{align*}

    进一步,我们可以来研究圆周运动。圆周运动的基本量是角度\(\theta\)、角速度\(\omega\)和角加速度\(\alpha\),
    这些物理量可以与上述的三个物理量相对应,其数学关系也是一致的,这里不再赘述。

\subsection{相对运动与参考系}
    一般我们在研究运动问题时,都是描述物体相对于大地的运动,这种方式可以很清楚地描述每个物体的具体运动方式。
    有时候我们可以用相对运动的方式去分析,这可以让我们把注意力放在需要被研究
    的对象上,而不用去考虑其他物体的运动,使得问题更加简洁。

    相对运动的理论基础是{\bfseries 伽利略变换},实际上,当我们说物体A相对于物体B的运动时,就是将物体A的参考系从大地参考系S,
    变换为了物体B的参考系\(S^\prime\)。

    \textbf{例2.2.1}:一个生活中常见的问题,在雨天中,经过相同的距离,是跑着淋雨更多,还是走着淋雨更多?(假设没有风,人可以抽象为立方体)

    人(作为一个立方体)在雨中前进,其运动方向前的那个面和上面是被雨淋的面。现在我们在雨的视角下分析(即找到了一个参考系\(S^{\prime}\),在这个参考系下雨是静止的),人在充满雨滴的空间中运动,
    其运动速度是向前的速度\(v_x\)和向上的速度\(v_y\)。两种情况下运动距离x一致,但水平速度\(v_x\)不同,因而运动的时间t不同。
    那么显然,正面淋到的雨的量不变,但是时间t的长短会影响上面淋到的雨的多少。t越小,淋到的雨越少。

\subsection{运动的分解}
    运动的分解是一种还原论的思想体现,它将复杂的问题简化为我们熟知的问题,这种还原论的思想贯彻了物理学的研究,
    我们熟知的一些叠加性原理,函数的级数展开等都是这种思想的体现。最早的时候,欧洲哲学家在完善地心说模型时
    提出了“本轮”、“均轮”模型,将复杂的天体运动还原为简单的圆周运动的叠加,这背后就是运动分解的思想。

    下面是几个常见的分解方法:

    \textbf{直角坐标系分解}:将运动分解为互相垂直方向的运动的叠加。我们熟悉的平抛运动等就是采用了这种方式,例如对其运动分析:

    \begin{minipage}{1\textwidth}
        \centering
        \begin{minipage}{0.2\textwidth}
            \centering
            \[
            \begin{cases}
                v_x = v_0 \\
                v_y = gt
            \end{cases}
            \]
        \end{minipage}
        \begin{minipage}{0.2\textwidth}
            \centering
            \[
            \begin{cases}
                s_x = v_x t \\
                s_y = \frac{1}{2} g t^2
            \end{cases}
            \]
        \end{minipage}
    \end{minipage}
    
    \ 

    \textbf{自然坐标系分解}:将物体的运动按照运动轨迹的法向和切向分解,
    一般来说,法向分量用角标n,切向分量用角标t。这种分解方式在研究曲线运动的物体时常用,
    特别是研究物体的瞬时运动特点时。自然坐标系中,任意一点运动满足:
    \begin{equation*}
        a_n = \frac{v^2}{\rho}
    \end{equation*}
    其中\(\rho\)是这一点的曲率半径,这是一个只与运动轨迹有关的量。该方程建立了任何一点法向加速度与速度之间的关系。
    特别的,在圆周运动中,将曲率半径替换为圆周运动半径,就得到了向心加速的计算公式。
    
    \textbf{运动种类分解}:用不同的运动方式来进行运动分解。例如,带电粒子在磁场中的螺线运动,可以分解为匀速运动和圆周运动
    垂直方向的叠加,滚动的轮子上一个点的运动(摆线运动)可以视为匀速运动和圆周运动在水平方向上叠加。

    \textbf{例2.3.1}:一个轮子在地面上无摩擦地滚动,轮子中心点水平速度为\(v_0\),轮子半径r。求这个轮子边缘上一点的运动轨迹。

    这里将轮子边缘上一点的运动分解为水平方向匀速运动和匀速圆周两个运动。水平运动速度\(v_0\),圆周运动有\(\omega = v_0/r\)。
    这两者叠加即可求出速度与时间的关系
    \begin{equation*}
        v_x = v_0 + v_0 \sin(\frac{v_0}{r}t + \varphi) \ \ \ \ \ \ v_y = v_0 \cos(\frac{v_0}{r}t + \varphi)
    \end{equation*}
    对上式积分可得路程随时间的关系,也就是在参数t下表示运动轨迹。其中\(\varphi\)是这一点的初相位。
    \[
    \begin{cases}
        s_x = v_0 t - r \cos(\dfrac{v_0}{r}t + \varphi) \\
        s_y = r \sin(\dfrac{v_0}{r}t + \varphi)
    \end{cases}
    \]
    
\section{质点动力学}

\subsection{牛顿运动定律}
    本小节我们来系统的讨论一下牛顿运动定律。以下是我们高中阶段所熟知的牛顿三定律的描述。

    \textbf{牛顿第一定律}:任何物体在不受力的状态下,会保持静止或匀速直线运动。

    \textbf{牛顿第二定律}:物体运动的加速度,与物体受到的力方向相同,与力的大小成正比,与物体质量成反比。

    \textbf{牛顿第三定律}:施力物体会受到受力物体的一个反作用力,这两个力大小相同,方向相反。

    这里我们重点关注以下第一和第二定律。首先,第一定律不能简单认为是第二定律的一个特殊情况,实际上,
    第一定律定义了一种参考系,在这个参考系下,第一第二定律才成立。我们把符合牛顿第一定律的参考系叫做
    \textbf{惯性参考系},简称惯性系。

    对于牛顿第二定律,有以下几个重要性质:
    \begin{itemize}[itemsep=-5pt]
        \item \textbf{瞬时性}:加速度的产生和消失是瞬时的。
        \item \textbf{独立性}:加速度与力的关系是一一对应的,每一个力的作用都会产生一个加速度。
        \item \textbf{可加性}:力产生的加速度是可线性叠加的,其叠加满足矢量加法。
    \end{itemize}
    利用独立性和可加性的原理,我们可以推导出一个质点系统的牛顿第二定律。假设一个系统由n个质点组成,
    每个质点受到了其他n-1个质点的力(系统内力)和一个外力,第i个质点的质量为\(m_i\)。整个系统的总受力为
    \begin{align*}
        \sum_i^n m_i a_i =& \sum_i^n (F_i +  \sum_{j,j\ne i}^{n}f_{ij}) \\
                         =& \sum_i^n F_i + \sum_i^n \sum_{j,j\ne i}^{n}f_{ij}
    \end{align*}
    其中\(F_i\)是第i个物体受到的外力,\(f_{ij}\)是第j个质点给第i个质点的内力。由牛顿第三定律可知,
    \(f_{ij} + f_{ji} = 0\),可以得到\(\sum_i^n \sum_{j,j\ne i}^{n}f_{ij} = 0\),带入可以得到对系统的牛顿第二定律:
    \begin{equation*}
        \sum_i^n m_i a_i = \sum_i^n F_i
    \end{equation*}
    这里的数学推导看上去十分显然且无聊,但是结论却十分有用,它告诉我们质点系的整体运动只与合外力有关。
    在具体的问题分析中,可以用这个结论,忽视系统内复杂的内力,只考虑外力的影响。

    \textbf{例3.1.1}:在地面上固定了一个斜面,斜面的倾角是\(\alpha\),表面的动摩擦系数是\(\mu\),
    在斜面上有一个质量为m的物体从静止开始下滑,求地面对斜面的合力的大小。

    \subsection{惯性系与非惯性系}
    在牛顿定律部分,我们利用牛顿第一定律定义了惯性参考系,惯性系是一种特殊的参考系,只有在这个参考系下牛顿运动定律才成立,
    我们熟知的动量定理和动能定理等都是在这个参考系下定义的。

    当然,惯性参考系也是一个理想的概念,我们现实中见到的参考系都是有加速的的(相对于一个理想的惯性系),
    在这样的参考系中,牛顿运动定理不再成立,这种参考系被称为\textbf{非惯性参考系}。在研究非惯性系时,
    可以引入一个\textbf{惯性力}来对运动进行修正。考虑一个有加速度a的非惯性系,在这个参考系下分析物体的运动,
    需要再考虑一个-a的加速度,由牛顿第二定律可知,这个加速度可以视为是由一个力\(f_{\text{惯}}\)引起的,对于质量为m的物体,这个力满足
    \begin{equation*}
        f_{\text{惯}} = -ma
    \end{equation*}
    需要注意的是,惯性力并不是真正的力,因为这个力并不存在一个施力物体,他只是我们在数学上引入的一个量,
    由此来让非惯性系也可以满足牛顿第二定律。

    上面由有平动加速度的参考系给出了惯性力的概念,而在旋转参考系中,惯性力主要是以\textbf{离心力}和\textbf{科里奥利力}(科氏力)
    的形式呈现。在一个以角速度\(\omega\)转动的参考系中,对物体受力分析都需要额外考虑一个离心力,离心力的大小满足
    \begin{equation*}
        f_{\text{离}} = m \omega^2 r
    \end{equation*}
    其中m是物体的质量,r是物理距离转动轴的距离,力的方向背离转动中心。如果转动参考系中的物体还有一个速度\(\vec{v}\),
    那么其在转动系中将受到一个科氏力,科氏力的大小和方向满足
    \begin{equation*}
        f_{\text{科}} = 2m\vec{v} \times \vec{\omega}
    \end{equation*}
    因为科氏力的方向与物体运动的方向垂直,所以会造成物体运动方向的偏转。地理学中用这个来解释河流对两岸侵蚀程度的差异,由于地球自传,
    其可以被视为一个转动参考系,以黄河为例,其运动方向是自西向东,所以会受到一个向南的科氏力,导致相同条件下河水对南岸的侵蚀程度比北岸强。

    平动加速度导致的惯性力和离心力都可以被视为保守力
    \footnote{如果一个力的做功与物体运动的轨迹无关,而只与其初始与末尾位置有关,则这个力被称为保守力,重力,电场力都是保守力。对于保守力,
    都可以引入一个“势”的概念去研究力的做工,下面将要提到的“离心势能”就是利用了这个性质。如果力的做功与物体运动轨迹有关,则称其为非保守力,如摩擦力。}
    ,我们可以引入对应的势能来研究。对于平动加速度产生的惯性力,其表现的效果可以等效为
    一种特殊的重力,所以可以用类似重力场的方式来研究,很多时候可以将该势能场与重力场叠加。

    \textbf{例3.2.1}:在一个水平面上以加速度a做匀加速运动的小车中,有一根刚性绳子连接车顶和一个质量为m的小球,绳子长度为l,求小球小角度摆动的周期。

    小车的参考系是非惯性系,在车中的小球m受到一个大小为ma,水平方向的惯性力。惯性力和重力组合形成了一个场,这个场的等效“重力加速度”大小为\(g^\prime = \sqrt{g^2+a^2}\),
    在这个等效场中,利用单摆周期公式可知\(T = 2\pi\sqrt{\dfrac{l}{\sqrt{g^2+a^2}}}\)

    离心力也是一种保守力,我们可以引入离心势能的概念,假设将旋转轴的位置(r=0处)的离心势能定义为0,则半径为r处的离心势能为
    \begin{align*}
        E &= -\int_0^r m\omega^2r \, dr  \\
          &= -\frac{1}{2}m\omega^2r^2
    \end{align*}
    在转动的非惯性系中,可以通过引入离心势能来保证能量守恒定律的成立。

    \textbf{例3.2.2}:在x-y坐标中有一条曲线,在这个曲线上串了一个质量m的小球(想象一下糖葫芦,只不过杆是弯的),小球与曲线之间无摩擦,可以在曲线的位置上
    自由活动,现在曲线杆以y轴为旋转轴,角速度\(\omega\)旋转。此时,小球在杆上任何一个位置都是随遇平衡状态\footnote{有关平衡状态的种类请参考},求曲线的表达式。

    在杆的旋转参考系中,小球受到重力和离心力的作用,由于是随遇平衡,任何位置的重力势能和离心势能之和都相等。我们不妨假设曲线经过原点O,则小球在原点位置的势能之和为0。
    则对任意位置都满足
    \begin{align*}
        mgy &= \frac{1}{2} m\omega^2 x^2 \\
        y &= \frac{\omega^2}{2g}x^2
    \end{align*}

    \subsection{简谐振动}
    高中课本中,我们对于简谐振动的认识来自于弹簧振子,我们定义,如果一个物体受到的力大小和离开平衡位置的距离成正比,
    方向始终指向平衡位置,那么这个物体将做简谐运动,用数学的形式可描述为,简谐振动满足的微分方程是
    \footnote{物理中常在一个变量上加点“\(\cdot\)”来表示对时间求导,点的数量表示求导的阶数}
    \begin{equation*}
        m\ddot{x} = -kx
    \end{equation*}
    之后,高中课本中直接给出了简谐振动的表达式,实际上,那个结果就是通过解这个常微分方程得出的,关于常微分方程的解法,
    大家可以参阅相关的高等数学的书籍,此处不再赘述。假设我们知道了简谐运动的初位置和初速度,就可以解方程,得到简谐振动表达式
    \begin{equation*}
        x = A\cos(\omega x + \varphi)
    \end{equation*}
    其中A表示振幅,\(\varphi\)是初相位,这两个量都是通过简谐振动的初始条件确定的。\(\omega\)是角速度,满足\(\omega = \sqrt{k/m}\)。

    以上是从受力的角度给出了简谐振动的定义。我们也可以从能量的角度来定义简谐振动,由于简谐振动的能量由动能和势能组成,
    在理想情况下也不会发生能量损失,于是简谐运动在位移x出的能量满足
    \begin{equation*}
        E = \frac{1}{2}m^* \dot{x}^{*2} + \frac{1}{2}kx^{*2}
    \end{equation*}
    这里k是一个常量,\(m^*\)和\(x^*\)表示广义质量和广义位移,也就是说,在具体的问题中,他们两个可以被替换为具有同等性质的物理量,
    例如在转动物体的简谐振动问题中,\(m^*\)可以表示转动惯量,而\(x^*\)表示角度。

    如果运动物体的能量满足上式,那么这个物体就做简谐振动,并同样满足角速度\(\omega = \sqrt{k/m^*}\)。对于一些受力复杂的系统,
    可以通过能量特征来判断物体是否是简谐振动,并求出其振动周期。

    


    \subsection{角动量与角动量守恒}
    角动量是描述转动物体时很重要的物理量,在质点动力学部分,我们主要关注一个或多个质点的角动量问题,
    在下一章会讨论一般刚体的角动量问题。

    质点运动的角动量定义为:若位矢为\(\vec{r}\)质量为m的质点以速度\(\vec{v}\)运动,则质点相对于原点的角动量是
    \begin{equation*}
        \vec{L} = \vec{r} \times \vec{p} = m \vec{r} \times \vec{v}
    \end{equation*}
    角动量可以理解为描述转动物体的动量,类似于平动物体的动量,对于角动量也有角动量定理和守恒定理。
    不同的是,在角动量定理中,对质点角动量产生影响的是力矩。在上述问题中,如果质点受到了一个\(\vec{F}\)
    的力,那么质点受到的力矩是
    \begin{equation*}
        \vec{N} = \vec{r} \times \vec{F}
    \end{equation*}
    有了力矩的概念,我们可以类比平动的角动量定理,得出\textbf{角动量定理}:
    \begin{equation*}
        \vec{N} = \frac{d\vec{L}}{dt}
    \end{equation*}
    类似平动中的动量守恒,运动的物体也有\textbf{角动量守恒原理}:质点或质点系在不受外力矩的条件下总角动量不变。
    
    很多时候在研究物体运动时会缺少方程,这时候可以考虑从角动量定理或角动量守恒出发列方程,这往往是解决问题的关键。



\section{刚体与刚体动力学}
    在高中阶段,大家研究的最多的物理模型是质点。质点是一个没有大小和形状的点,


\end{document}
