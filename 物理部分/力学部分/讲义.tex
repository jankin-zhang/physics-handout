\documentclass{article}
\usepackage[utf8]{inputenc}
\usepackage[UTF8]{ctex}
\usepackage{geometry}
\usepackage{graphics}
\usepackage{float}
\usepackage{adjustbox}
\usepackage{fancyhdr}
\usepackage{titletoc}
\usepackage{amsmath}
\usepackage{enumitem}
\usepackage{multirow}
\usepackage{wrapfig}
\usepackage[colorlinks=false]{hyperref}
\usepackage[labelfont=bf]{caption}


\title{\bfseries \Huge 力学讲义}  %题目
\author{bilibili:德布罗意不是波}  %作者
\date{}  %日期
\fancyhf{}  
\fancyhead[C]{力学讲义}  %页眉
\fancyfoot[C]{\thepage}  %页脚(设定页码)
\pagestyle{fancy}

%设定目录格式
\titlecontents{chapter}
  [1em]
  {\bfseries\huge}
  {\contentslabel{1em}} 
  {\hspace*{0em}}
  {\titlerule*[0.9pc]{.}\contentspage}

\titlecontents{section}
  [2.5em] 
  {\bfseries\large}
  {\contentslabel{2.5em}} 
  {\hspace*{0em}}
  {\titlerule*[0.45pc]{.}\dotfill\contentspage}

\titlecontents{subsection}
  [4.1em] 
  {}
  {\contentslabel{3.5em}} 
  {\hspace*{0em}}
  {\titlerule*[0.44pc]{.}\dotfill\contentspage}

\hypersetup{
    hidelinks,         
}

\begin{document}
  
\maketitle
\tableofcontents
\newpage

\section{数学知识}
    力学部分所需的数学知识并不复杂,一般而言,只要是在大学学过一学期的高等数学的课程,就可以
    轻松应对本部分出现的数学推导。对于高中生而言,也可以通过简单地接触一些高等数学的概念来快速上手。
    由于本讲义主要是以物理学概念的讲述为主,并不会很系统的涉及到数学内容,所以需要读者对以下知识有所了解。

    1.极限相关知识:包括简单的极限计算、泰勒展开公式等。

    2.微分、积分运算:包括熟悉导数和微分、能利用牛顿-莱布尼茨公式计算积分等。

    3.常微分方程:能用分离变量的方法解决简单的常微分方程。

    4.以及对未知事物的好奇心!

\section{质点动理学}
    质点是我们很熟悉的一个概念,高中物理中的绝大部分力学问题都是解决一个质点的运动。在本部分的内容中,
    我们将在高中学习内容的基础上,构建一个更加完善的理论体系,并补充一些物理的分析方法。

\subsection{运动的基本问题}
    在坐标系中描述一个质点的运动,我们常用的物理量是时间t、位置矢量\(\vec{r}\)、位移\(\vec{x}\)、速度\(\vec{v}\)和加速度\(\vec{a}\)。
    这些物理量之间的关系如下:
    \begin{align*}
        \vec{x} = \varDelta \vec{r}\ \ \  \vec{v}  = &\frac{d\vec{x}}{dt}\ \ \ \vec{a} = \frac{d\vec{v}}{dt} \\
        \vec{x} = x_0 + \int \vec{v} \,dt & \ \ \ \vec{v} = v_0 + \int \vec{a} \,dt 
    \end{align*}
    通常对于一个简单的直线运动,我们可以很容易地利用上述公式解决质点的运动问题。

    \textbf{例2.1.1}:一个质点在x轴上沿着正方向运动,其运动速度与时间的关系为\(v(t)=t+\sin t\),那么从t=0到\(t=\pi\)的时间内,
    质点运动的距离是多少?
    \begin{align*}
        \varDelta x = \int_0^{\pi} t+\sin t \,dt \\
        \varDelta x = \frac{1}{2}\pi^2 + 2
    \end{align*}

    进一步,我们可以来研究圆周运动。圆周运动的基本量是角度\(\theta\)、角速度\(\omega\)和角加速度\(\alpha\),
    这些物理量可以与上述的三个物理量相对应,其数学关系也是一致的,这里不再赘述。

\subsection{相对运动与参考系}
    一般我们在研究运动问题时,都是描述物体相对于大地的运动,这种方式可以很清楚地描述每个物体的具体运动方式。
    有时候我们可以用相对运动的方式去分析,这可以让我们把注意力放在需要被研究
    的对象上,而不用去考虑其他物体的运动,使得问题更加简洁。

    相对运动的理论基础是{\bfseries 伽利略变换},实际上,当我们说物体A相对于物体B的运动时,就是将物体A的参考系从大地参考系S,
    变换为了物体B的参考系\(S^\prime\)。

    \textbf{例2.2.1}:一个生活中常见的问题,在雨天中,经过相同的距离,是跑着淋雨更多,还是走着淋雨更多?(假设没有风,人可以抽象为立方体)

    人(作为一个立方体)在雨中前进,其运动方向前的那个面和上面是被雨淋的面。现在我们在雨的视角下分析(即找到了一个参考系\(S^{\prime}\),在这个参考系下雨是静止的),人在充满雨滴的空间中运动,
    其运动速度是向前的速度\(v_x\)和向上的速度\(v_y\)。两种情况下运动距离x一致,但水平速度\(v_x\)不同,因而运动的时间t不同。
    那么显然,正面淋到的雨的量不变,但是时间t的长短会影响上面淋到的雨的多少。t越小,淋到的雨越少。

\subsection{运动的分解}
    运动的分解是一种还原论的思想体现,它将复杂的问题简化为我们熟知的问题,这种还原论的思想贯彻了物理学的研究,
    我们熟知的一些叠加性原理,函数的级数展开等都是这种思想的体现。最早的时候,欧洲哲学家在完善地心说模型时
    提出了“本轮”、“均轮”模型,将复杂的天体运动还原为简单的圆周运动的叠加,这背后就是运动分解的思想。

    下面是几个常见的分解方法:

    \textbf{直角坐标系分解}:将运动分解为互相垂直方向的运动的叠加。我们熟悉的平抛运动等就是采用了这种方式,例如对其运动分析:

    \begin{minipage}{1\textwidth}
        \centering
        \begin{minipage}{0.2\textwidth}
            \centering
            \[
            \begin{cases}
                v_x = v_0 \\
                v_y = gt
            \end{cases}
            \]
        \end{minipage}
        \begin{minipage}{0.2\textwidth}
            \centering
            \[
            \begin{cases}
                s_x = v_x t \\
                s_y = \frac{1}{2} g t^2
            \end{cases}
            \]
        \end{minipage}
    \end{minipage}
    
    \ 

    \textbf{自然坐标系分解}:将物体的运动按照运动轨迹的法向和切向分解,
    一般来说,法向分量用角标n,切向分量用角标t。这种分解方式在研究曲线运动的物体时常用,
    特别是研究物体的瞬时运动特点时。自然坐标系中,任意一点运动满足:
    \begin{equation*}
        a_n = \frac{v^2}{\rho}
    \end{equation*}
    其中\(\rho\)是这一点的曲率半径,这是一个只与运动轨迹有关的量。该方程建立了任何一点法向加速度与速度之间的关系。
    特别的,在圆周运动中,将曲率半径替换为圆周运动半径,就得到了向心加速的计算公式。
    
    \textbf{运动种类分解}:用不同的运动方式来进行运动分解。例如,带电粒子在磁场中的螺线运动,可以分解为匀速运动和圆周运动
    垂直方向的叠加,滚动的轮子上一个点的运动(摆线运动)可以视为匀速运动和圆周运动在水平方向上叠加。

    \textbf{例2.3.1}:一个轮子在地面上无摩擦地滚动,轮子中心点水平速度为\(v_0\),轮子半径r。求这个轮子边缘上一点的运动轨迹。

    这里将轮子边缘上一点的运动分解为水平方向匀速运动和匀速圆周两个运动。水平运动速度\(v_0\),圆周运动有\(\omega = v_0/r\)。
    这两者叠加即可求出速度与时间的关系
    \begin{equation*}
        v_x = v_0 + v_0 \sin(\frac{v_0}{r}t + \varphi) \ \ \ \ \ \ v_y = v_0 \cos(\frac{v_0}{r}t + \varphi)
    \end{equation*}
    对上式积分可得路程随时间的关系,也就是在参数t下表示运动轨迹。其中\(\varphi\)是这一点的初相位。
    \[
    \begin{cases}
        s_x = v_0 t - r \cos(\dfrac{v_0}{r}t + \varphi) \\
        s_y = r \sin(\dfrac{v_0}{r}t + \varphi)
    \end{cases}
    \]
    
\section{质点动力学}

    \subsection{牛顿运动定律}
    本小节我们来系统的讨论一下牛顿运动定律。以下是我们高中阶段所熟知的牛顿三定律的描述。

    \textbf{牛顿第一定律}:任何物体在不受力的状态下,会保持静止或匀速直线运动。

    \textbf{牛顿第二定律}:物体运动的加速度,与物体受到的力方向相同,与力的大小成正比,与物体质量成反比。

    \textbf{牛顿第三定律}:施力物体会受到受力物体的一个反作用力,这两个力大小相同,方向相反。

    这里我们重点关注以下第一和第二定律。首先,第一定律不能简单认为是第二定律的一个特殊情况,实际上,
    第一定律定义了一种参考系,在这个参考系下,第一第二定律才成立。我们把符合牛顿第一定律的参考系叫做
    \textbf{惯性参考系},简称惯性系。

    对于牛顿第二定律,有以下几个重要性质:
    \begin{itemize}[itemsep=-5pt]
        \item \textbf{瞬时性}:加速度的产生和消失是瞬时的。
        \item \textbf{独立性}:加速度与力的关系是一一对应的,每一个力的作用都会产生一个加速度。
        \item \textbf{可加性}:力产生的加速度是可线性叠加的,其叠加满足矢量加法。
    \end{itemize}
    利用独立性和可加性的原理,我们可以推导出一个质点系统的牛顿第二定律。假设一个系统由n个质点组成,
    每个质点受到了其他n-1个质点的力(系统内力)和一个外力,第i个质点的质量为\(m_i\)。整个系统的总受力为
    \begin{align*}
        \sum_i^n m_i a_i =& \sum_i^n (F_i +  \sum_{j,j\ne i}^{n}f_{ij}) \\
                         =& \sum_i^n F_i + \sum_i^n \sum_{j,j\ne i}^{n}f_{ij}
    \end{align*}
    其中\(F_i\)是第i个物体受到的外力,\(f_{ij}\)是第j个质点给第i个质点的内力。由牛顿第三定律可知,
    \(f_{ij} + f_{ji} = 0\),可以得到\(\displaystyle \sum_{i}^{n} \sum_{j,j\ne i}^{n}f_{ij} = 0\),带入可以得到对系统的牛顿第二定律:
    \begin{equation*}
        \sum_i^n m_i a_i = \sum_i^n F_i
    \end{equation*}
    这里的数学推导看上去十分显然且无聊,但是结论却十分有用,它告诉我们质点系的整体运动只与合外力有关。
    在具体的问题分析中,可以用这个结论,忽视系统内复杂的内力,只考虑外力的影响。

    \textbf{例3.1.1}:在地面上固定了一个斜面,斜面的倾角是\(\alpha\),表面的动摩擦系数是\(\mu\),
    在斜面上有一个质量为m的物体从静止开始下滑,求地面对斜面的合力的大小。

    \subsection{惯性系与非惯性系}
    在牛顿定律部分,我们利用牛顿第一定律定义了惯性参考系,惯性系是一种特殊的参考系,只有在这个参考系下牛顿运动定律才成立,
    我们熟知的动量定理和动能定理等都是在这个参考系下定义的。

    当然,惯性参考系也是一个理想的概念,我们现实中见到的参考系都是有加速的的(相对于一个理想的惯性系),
    在这样的参考系中,牛顿运动定理不再成立,这种参考系被称为\textbf{非惯性参考系}。在研究非惯性系时,
    可以引入一个\textbf{惯性力}来对运动进行修正。考虑一个有加速度a的非惯性系,在这个参考系下分析物体的运动,
    需要再考虑一个-a的加速度,由牛顿第二定律可知,这个加速度可以视为是由一个力\(f_{\text{惯}}\)引起的,对于质量为m的物体,这个力满足
    \begin{equation*}
        f_{\text{惯}} = -ma
    \end{equation*}
    需要注意的是,惯性力并不是真正的力,因为这个力并不存在一个施力物体,他只是我们在数学上引入的一个量,
    由此来让非惯性系也可以满足牛顿第二定律。

    上面由有平动加速度的参考系给出了惯性力的概念,而在旋转参考系中,惯性力主要是以\textbf{离心力}和\textbf{科里奥利力}(科氏力)
    的形式呈现。在一个以角速度\(\omega\)转动的参考系中,对物体受力分析都需要额外考虑一个离心力,离心力的大小满足
    \begin{equation*}
        f_{\text{离}} = m \omega^2 r
    \end{equation*}
    其中m是物体的质量,r是物理距离转动轴的距离,力的方向背离转动中心。如果转动参考系中的物体还有一个速度\(\vec{v}\),
    那么其在转动系中将受到一个科氏力,科氏力的大小和方向满足
    \begin{equation*}
        f_{\text{科}} = 2m\vec{v} \times \vec{\omega}
    \end{equation*}
    因为科氏力的方向与物体运动的方向垂直,所以会造成物体运动方向的偏转。地理学中用这个来解释河流对两岸侵蚀程度的差异,由于地球自传,
    其可以被视为一个转动参考系,以黄河为例,其运动方向是自西向东,所以会受到一个向南的科氏力,导致相同条件下河水对南岸的侵蚀程度比北岸强。

    平动加速度导致的惯性力和离心力都可以被视为保守力
    \footnote{如果一个力的做功与物体运动的轨迹无关,而只与其初始与末尾位置有关,则这个力被称为保守力,重力,电场力都是保守力。对于保守力,
    都可以引入一个“势”的概念去研究力的做工,下面将要提到的“离心势能”就是利用了这个性质。如果力的做功与物体运动轨迹有关,则称其为非保守力,如摩擦力。}
    ,我们可以引入对应的势能来研究。对于平动加速度产生的惯性力,其表现的效果可以等效为
    一种特殊的重力,所以可以用类似重力场的方式来研究,很多时候可以将该势能场与重力场叠加。

    \textbf{例3.2.1}:在一个水平面上以加速度a做匀加速运动的小车中,有一根刚性绳子连接车顶和一个质量为m的小球,绳子长度为l,求小球小角度摆动的周期。

    小车的参考系是非惯性系,在车中的小球m受到一个大小为ma,水平方向的惯性力。惯性力和重力组合形成了一个场,这个场的等效“重力加速度”大小为\(g^\prime = \sqrt{g^2+a^2}\),
    在这个等效场中,利用单摆周期公式可知\(T = 2\pi\sqrt{\dfrac{l}{\sqrt{g^2+a^2}}}\)

    离心力也是一种保守力,我们可以引入离心势能的概念,假设将旋转轴的位置(r=0处)的离心势能定义为0,则半径为r处的离心势能为
    \begin{align*}
        E &= -\int_0^r m\omega^2r \, dr  \\
          &= -\frac{1}{2}m\omega^2r^2
    \end{align*}
    在转动的非惯性系中,可以通过引入离心势能来保证能量守恒定律的成立。

    \textbf{例3.2.2}:在x-y坐标中有一条曲线,在这个曲线上串了一个质量m的小球(想象一下糖葫芦,只不过杆是弯的),小球与曲线之间无摩擦,可以在曲线的位置上
    自由活动,现在曲线杆以y轴为旋转轴,角速度\(\omega\)旋转。此时,小球在杆上任何一个位置都是随遇平衡状态\footnote{有关平衡状态的种类请参考4.3},求曲线的表达式。

    在杆的旋转参考系中,小球受到重力和离心力的作用,由于是随遇平衡,任何位置的重力势能和离心势能之和都相等。我们不妨假设曲线经过原点O,则小球在原点位置的势能之和为0。
    则对任意位置都满足
    \begin{align*}
        mgy &= \frac{1}{2} m\omega^2 x^2 \\
        y &= \frac{\omega^2}{2g}x^2
    \end{align*}

    \subsection{动量与动量守恒}
    本讲义在力学部分只探讨了动量内容,忽视了能量内容,主要是因为力学部分的能量和势的内容相对单调,不如将其完全放到电磁学中。

    高中阶段,我们对于动量的认识很多时候都来源于小球碰撞,并在该模型中认识了动量定理和动量守恒。现在让我们再回归到最开始的小球碰撞问题,
    假设有两个小球,质量和速度分别为\(m_1\),\(v_1\)和\(m_2\),\(v_2\)。在某一时刻二者相撞,之后二者的速度分别为\(v_1^\prime\)和\(v_2^\prime\)。
    如果是完全弹性碰撞,那么我们可以根据动量守恒和能量守恒得到
    \begin{align*}
        m_1 v_1 + m_2 v_2 &= m_1 v_1^\prime + m_2 v_2\prime \\
        \frac{1}{2}m_1 v_1^2 + \frac{1}{2} m_2 v_2^2 &= \frac{1}{2} m_1 v_1^{\prime 2} + \frac{1}{2} m_2 v_2^{\prime 2}
    \end{align*}
    我们将这两个式子改写为
    \begin{align*}
        m_1(v_1^\prime - v_1) = m_2(v_2 - v_2^\prime) \ , \  m_1(v_1^{\prime 2} - v_1^2) = m_2(v_2^2 - v_2^{\prime 2})
    \end{align*}
    接着将两个式子相除,然后移项可得
    \begin{equation*}
        v_2 - v_1 = v_1^\prime - v_2^\prime
    \end{equation*}
    在这里,方程的左边可以视为碰撞之前,两个球相互靠近的相对速度,方程右边是碰撞之后,两球相互远离的相对速度。于是,我们得到了一个很重要的结论——
    对于完全弹性碰撞,碰撞前后两球的相对速度的大小不变。

    对于一般的碰撞,碰撞之后的相对速度是小于碰撞之前的,特别的,完全非弹性碰撞的碰撞后相对速度是0。我们定义恢复系数\(e = \dfrac{|v_1^\prime - v_2^\prime|}{|v_1 - v_2|}\)
    来描述碰撞过后相对速度的变化。于是,对于完全弹性碰撞,\(e=1\)。对于完全非弹性碰撞,\(e=0\)。且所有的碰撞都满足\(0\le e \le 1\)。

    恢复系数实质上是能量守恒定律的一种体现,恢复系数与1的差值表示碰撞过程中损失能量的多少。在实际的问题中,我们可以用恢复系数满足的公式代替能量守恒的公式,
    与动量守恒公式联立,从而更简洁地得出碰撞问题的结果。也可以利用恢复系数的取值范围,判断一个碰撞过程是否会发生。

    \textbf{例3.3.1}:两个小球发生一维碰撞,假设二者质量和初速度分别为\(m_1 = 1kg\),\(v_1 = 5m/s\)和\(m_2 =1kg\),\(v_2 = 2m/s\)。则下面四个选项中的碰撞后速度,不可能出现的是
    \begin{align*}
        A.v_1^\prime = 3m/s , v_2^\prime = 4m/s \ \ \ \ \ \ B.v_1^\prime =  3m/s , v_2^\prime = 5m/s \\
        C.v_1^\prime = 2m/s , v_2^\prime = 5m/s \ \ \ \ \ \ D.v_1^\prime =  1m/s , v_2^\prime = 6m/s
    \end{align*}
    对于这种题目,可以直接使用恢复系数取值范围和动量守恒两个方程判断,只有二者都满足的才是可能发生的结果。这里注意到B不满足动量守恒,D的恢复系数大于1,故选BD。

    我们将只有两个质点,且只存在两个质点之间相互作用力的系统的问题称为\textbf{二体问题}。显然,在二体问题中系统的质心位置不变,所以在质心参考系下分析将会十分简洁。不过在这里我们采用另外一种分析方式,
    假设两个质点的质量分别是\(m_1\)和\(m_2\),相互作用力大小是F,那么二者的相对加速度\(a = F(\dfrac{1}{m_1} + \dfrac{1}{m_2})\),也就是说,在\(m_1\)参考系(非惯性系)下\(m_2\)的加速度就是\(a\)。
    于是在\(m_1\)参考系下,我们定义约化质量\(\mu = \dfrac{m_1 m_2}{m_1 + m_2}\),那么\(m_2\)满足
    \begin{equation*}
        F = \mu a
    \end{equation*}
    也就是说,在二体问题的非惯性系中,我们不需要引入惯性力,而是仅仅“修正”了物体的质量,就可以按照正常的牛顿第二定律分析物体的运动。这种分析方式在天体运动和弹簧连接的小球问题中经常使用。

    \textbf{例3.3.2}:假设有两个天体,质量分别为M和m,二者之间的距离是R,两个天体构成了一个稳定的系统,都围绕着连线上某一点做匀速圆周运动,求这个运动的周期。

    这里的”某一点“显然就是系统的质心,我们可以确定两天体到质心的距离,也就确定了各自圆周运动的半径,然后用圆周运动的规律求解。不过这里可以直接在M的参考系中分析m运动,设\(\mu = \dfrac{Mm}{M+m}\),有
    \begin{equation*}
        \frac{G m_1 m_2}{R^2} = \mu \frac{4\pi^2}{T^2} R
    \end{equation*}
    故运动的周期是\(T = 2\pi \sqrt{\dfrac{R^3}{G(m_1 + m_2)}}\)。

    现在我们用这种二体运动的方式考虑一下小球碰撞的问题。如果发生了完全非弹性碰撞,在一个小球的参考系中,相当于另一个球撞击这个球,然后动能完全损耗。用约化质量表示出的能量损耗为\(\frac{1}{2}\mu (v_1 - v_2)^2\)。
    而我们知道,完全非弹性碰撞是能量损耗最大的一种碰撞形式,也就是说,在这种情况下,小球的所有可被转化利用的动能全都被损耗了。我们将这个可被利用的能量称为\textbf{资用能},表示为\(E_x\),其满足
    \begin{equation*}
        E_x = \frac{1}{2}\mu v^2
    \end{equation*}
    其中\(v = |v_1 - v_2|\),是两个小球碰撞之前的相对速度。

    关于资用能,我们可以视为二体问题中用约化质量计算出的动能,也可以视为在质心参考系下的动能之和,读者可以自行证明,二者所表示出的资用能\(E_x\)是相等的。资用能在二体问题中十分有用,例如,在平面上两个被弹簧连接的小球的问题中,
    弹簧最大的压缩量实际上就是资用能完全转化为了弹性势能。在粒子对撞实验中,也会尽量增加粒子对撞的资用能,从而释放更大的能量。
    \footnote{热力学中,资用能是指相对于某一环境状态而言,系统能够做功的最大可用量}


    \subsection{简谐振动}
    高中课本中,我们对于简谐振动的认识来自于弹簧振子,我们定义,如果一个物体受到的力大小和离开平衡位置的距离成正比,
    方向始终指向平衡位置,那么这个物体将做简谐运动,用数学的形式可描述为,简谐振动满足的微分方程是
    \footnote{物理中常在一个变量上加点“\(\cdot\)”来表示对时间求导,点的数量表示求导的阶数}
    \begin{equation*}
        m\ddot{x} = -kx
    \end{equation*}
    之后,高中课本中直接给出了简谐振动的表达式,实际上,那个结果就是通过解这个常微分方程得出的,关于常微分方程的解法,
    大家可以参阅相关的高等数学的书籍,此处不再赘述。假设我们知道了简谐运动的初位置和初速度,就可以解方程,得到简谐振动表达式
    \begin{equation*}
        x = A\cos(\omega x + \varphi)
    \end{equation*}
    其中A表示振幅,\(\varphi\)是初相位,这两个量都是通过简谐振动的初始条件确定的。\(\omega\)是角速度,满足\(\omega = \sqrt{k/m}\)。

    以上是从受力的角度给出了简谐振动的定义。我们也可以从能量的角度来定义简谐振动,由于简谐振动的能量由动能和势能组成,
    在理想情况下也不会发生能量损失,于是简谐运动在位移x出的能量满足
    \begin{equation*}
        E = \frac{1}{2}m^* \dot{x}^{*2} + \frac{1}{2}kx^{*2}
    \end{equation*}
    这里k是一个常量,\(m^*\)和\(x^*\)表示广义质量和广义位移,也就是说,在具体的问题中,他们两个可以被替换为具有同等性质的物理量,
    例如在转动物体的简谐振动问题中,\(m^*\)可以表示转动惯量,而\(x^*\)表示角度。

    如果运动物体的能量满足上式,那么这个物体就做简谐振动,并同样满足角速度\(\omega = \sqrt{k/m^*}\)。对于一些受力复杂的系统,
    可以通过能量特征来判断物体是否是简谐振动,并求出其振动周期。

    \textbf{例3.4.1}:一个均匀圆盘被三根长为L的绳子悬挂,绳子的悬挂点在圆盘的边缘,且三个悬挂点将圆盘三等分。现在圆盘做小角度
    扭转振动,求振动的周期。(本题涉及到刚体的旋转,建议先学习刚体部分的内容)

    这里的受力情况不是很好判断,我们可以通过能量的角度去计算。这里的\(x^*\)为转过的角度\(\theta\),\(m^*\)是圆盘转动惯量I。
    振动的动能就是圆盘转动的动能,势能是转动过程中增加的重力势能。假设圆盘的质量为m,半径为r。在转过\(\theta\)角度后圆盘上升高度\(\Delta h = L - \sqrt{L^2 - \theta^2 r^2}\)。
    这里利用了小角度下\(\sin \theta = \theta \),故振动的能量满足
    \begin{equation*}
        E = \frac{1}{2} I \dot{\theta}^2 + mg(L-\sqrt{L^2 - \theta^2 r^2})
    \end{equation*}
    这里对根号的部分近似可得
    \footnote{这里的近似是,若\(x \ll 1\),有\(\sqrt{1+x} = 1+\dfrac{1}{2}x\)}
    \begin{equation*}
        E =  \frac{1}{2} \cdot \frac{1}{2}mr^2 \dot{\theta}^2 + \frac{1}{2}\frac{\theta^2 r^2}{L}mg 
    \end{equation*}
    这里可见常数k等效为\(mgr^2/L\),故可计算得
    \begin{align*}
        T &= 2\pi \sqrt{\frac{m^*}{k}} \\
          &= 2\pi \sqrt{\frac{L}{2g}}
    \end{align*}

    从上述内容中看出,简谐振动表达式是一个三角函数,所以我们可以参照三角函数的定义,将简谐振动看成是一个圆周运动到坐标轴上的投影。
    这里我们定义一个起点在坐标原点,长度为A的位置矢量(简称位矢),如果这个位矢的初相位是\(\varphi\),旋转角速度是\(\omega\),
    那么其顶点在x轴上的投影就是简谐运动\(x=A\cos(\omega x + \varphi)\)。因为这个位矢的运动是相对直观的圆周运动,且其在运算的过程中
    满足矢量的性质,所以很多时候可以通过研究位矢来了解一个简谐振动。

    \textbf{例3.4.2}:一个简谐振动的振幅是2,那么这个振动从x=2处向左第一次运动到x=1处,和从x=1处向左第一次运动到x=0处所用的时间的比值是多少

    x=0,1,2对应的位矢的角度分别是\(\pi/2\),\(\pi/3\)和0,因为位矢是匀速圆周运动,所以时间比和经过的角度比相同,可得是2:1

    接下来,我们利用位矢来研究一下简谐振动的叠加。这里我们所研究的振动都是角速度相同的,这样其位矢在运动过程中彼此的相位差将不变。
    假设有两个角速度相同的振动\(x_1 = A_1 \cos(\omega x +\varphi_1)\)和\(x_2 = A_2 \cos(\omega x + \varphi_2)\),计算其位矢相加,
    即计算\(\vec{A_1}+\vec{A_2}\),二者在x轴上的投影大小分别是\(A_1\cos\varphi_1\)和\(A_2\cos \varphi_2\),在y轴上的投影大小为
    \(A_1\sin \varphi_1\)和\(A_2 \sin \varphi_2\),可得合成的位矢大小
    \begin{align*}
        A^2 =& A_1^2\cos^2\varphi_1 + 2A_1A_2 \cos\varphi_1 \cos\varphi_2 + A_2^2 \cos^2 \varphi_2 \\
             & + A_1^2 \sin^2 \varphi_1 + 2A_1A_2 \sin\varphi_1\sin \varphi_2 + A_2^2 \sin^2 \varphi_2 \\
            =& A_1^2(\cos^2\varphi_1 + \sin^2\varphi_1) + A_2^2(\cos^2\varphi_2 + \sin^2\varphi_2) \\
             & + 2A_1A_2(\cos\varphi_1 \cos\varphi_2 + \sin\varphi_1 \sin\varphi_2)
    \end{align*}
    \begin{equation*}
        =A_1^2 + A_2^2 + 2A_1A_2 \cos(\varphi_2 - \varphi_1)
    \end{equation*}
    即得和振动的振幅
    \begin{equation*}
        A = \sqrt{A_1^2 + A_2^2 + 2A_1A_2 \cos(\varphi_2 - \varphi_1)}
    \end{equation*}
    而通过矢量加法的分析得到和振幅初相位
    \begin{equation*}
        \tan \varphi = \frac{A_1 \sin\varphi_1 + A_2 \sin\varphi_2}{A_1 \cos\varphi_1 + A_2\cos\varphi_2}
    \end{equation*}
    通过矢量加法性质和对结果的分析可知,当两个振动的初相位相差\(2k\pi\)(k是整数)时,和振动振幅最大,为\(A_1 + A_2\)。
    当初相位相差\((2k+1)\pi\)时,和振动的振幅最小,为\(|A_1 - A_2|\)。


    \subsection{角动量与角动量守恒}
    角动量是描述转动物体时很重要的物理量,在质点动力学部分,我们主要关注一个或多个质点的角动量问题,
    在下一章会讨论一般刚体的角动量问题。

    质点运动的角动量定义为:若位矢为\(\vec{r}\)质量为m的质点以速度\(\vec{v}\)运动,则质点相对于原点的角动量是
    \begin{equation*}
        \vec{L} = \vec{r} \times \vec{p} = m \vec{r} \times \vec{v}
    \end{equation*}
    角动量可以理解为描述转动物体的动量,类似于平动物体的动量,对于角动量也有角动量定理和守恒定理。
    不同的是,在角动量定理中,对质点角动量产生影响的是力矩。在上述问题中,如果质点受到了一个\(\vec{F}\)
    的力,那么质点受到的力矩是
    \begin{equation*}
        \vec{N} = \vec{r} \times \vec{F}
    \end{equation*}
    有了力矩的概念,我们可以类比平动的角动量定理,得出\textbf{角动量定理}:
    \begin{equation*}
        \vec{N} = \frac{d\vec{L}}{dt}
    \end{equation*}
    类似平动中的动量守恒,运动的物体也有\textbf{角动量守恒原理}:质点或质点系在不受外力矩的条件下总角动量不变。
    
    很多时候在研究物体运动时会缺少方程,这时候可以考虑从角动量定理或角动量守恒出发列方程,这往往是解决问题的关键。

    \textbf{例3.5.1}:地球绕太阳运动可近似为匀速圆周运动,假设其半径为R,角速度是\(\omega\)。有一时刻,一个小行星以速度v
    撞击地球,撞击时刻小行星的速度正好沿圆周运动的切线方向,且与地球运动方向相同。小行星质量m,地球质量为M,求撞击之后二者共同的速度

    对于小行星和地球的系统,整体不受外力矩,故总角动量不变
    \begin{align*}
        M \omega R^2 +& mvR = (M+m)v\prime R \\
        v\prime &= \frac{M\omega R + mv}{M+m}
    \end{align*}





\section{刚体与刚体动力学}
    在高中阶段,大家研究的最多的物理模型是质点。质点是一个没有大小和形状的点,所以使得模型的分析变得十分简洁。
    例如不需要考虑物体受力点的位置,也不需要考虑物体的旋转。而刚体是一个有大小和形状的物体,且其在运动过程中的
    大小和形状都不会发生变化。刚体的运动主要分为平动和定轴转动两种,平动是指刚体在运动过程中任意两点的连线始终保持平行,
    定轴转动即绕着一个固定的旋转轴旋转。

    \subsection{刚体的转动惯量}
    在上一章节我们初步接触了质点的角动量,而刚体可以视为是很多个质点的组合,所以可以使用求和或积分的形式去推导刚体的角动量。
    我们将一个刚体分为n个质点,且第i个质点的质量是\(m_i\),到达旋转轴的位矢是\(r_i\),那么假设刚体绕旋转轴(不妨设为z轴)以角速度\(\omega\)旋转,
    则刚体的角动量
    \footnote{下式当中的\(\hat{z}\)表示z方向的单位向量,物理中常用乘以单位向量表示向量的方向}
    \begin{align*}
        \vec{L} = \sum_i^n m_i r_i \times v_i \\
                = \sum_i^n (m_i r_i^2) \omega \hat{z}
    \end{align*}
    这里我们定义\(I =\displaystyle \sum_i^n m_i r_i^2\),则刚体的角动量可定义为
    \begin{equation*}
        L = I \omega
    \end{equation*}
    这里的I是一个只与刚体自身有关的量,它描述了质量相对于旋转轴的分布情况,我们将其称之为\textbf{转动惯量}。通过转动惯量的定义可见,
    只要确定了刚体和转动轴,就可以计算刚体相对于该转动轴的转动惯量。对于质量连续分布的刚体,可将其改为积分形式
    \begin{equation*}
        I = \int r^2 \, dm
    \end{equation*}
    转动惯量的计算需要找到m与r的关系,在计算过程中往往需要借助密度来实现。

    \textbf{例4.1.1}:求一个长l,质量m的杆的转动惯量,转动轴经过杆的一个顶点,与杆垂直。

    可假设杆的线密度为\(\lambda\),那么转动惯量的计算公式可以改写为
    \footnote{这里使用了微积分中的换元积分,由\(m=\lambda x\),微元\(dm\)可以视为\(dx\)长度的木棍的质量,即\(dm = \lambda dx\),完成换元}
    \begin{align*}
        I =& \int_0^l r^2 \lambda \, dr \\
          =& \frac{ml^2}{3}
    \end{align*}

    一些较为规则的几何体都可以直接积分算出其转动惯量,有时候也可以使用量纲法计算转动惯量。下面给出了一些常见的转动惯量
    \begin{table}[H]
        \centering
        \begin{tabular}{|c|c|c|}
        \hline
        \textbf{物体}                  & \textbf{转动轴} & \textbf{转动惯量} \\ \hline
        两端开通的薄圆柱壳,半径为r,质量m           & 圆柱的中心轴       &  $I = mr^2$             \\ \hline
        \multirow{2}{*}{实心圆柱,半径为r,高h,质量m}             & 圆柱中心轴        &   $ I=\dfrac{mr^2}{2}$          \\ \cline{2-3}
                                    & 过圆柱中心,与圆柱的轴垂直  &   $ I=\dfrac{1}{12}m(3r^2 + h^2) $   \\ \hline
        \multirow{2}{*}{薄圆盘,半径r,质量m} & 过圆盘中心,垂直于圆盘  &    $ I=\dfrac{mr^2}{2}$           \\ \cline{2-3} 
                                     & 过圆盘中心,平行于圆盘  &       $I=\dfrac{mr^2}{4}$        \\ \hline
        \multirow{2}{*}{圆环,半径为r,质量m} & 过圆环中心,垂直与圆环面 &        $I=mr^2$       \\ \cline{2-3} 
                                     & 过圆环中心,平行于圆环面 &      $I=\dfrac{mr^2}{2}$         \\ \hline
        实心球,半径r,质量m                  & 过球心          &     $I=\dfrac{2mr^2}{5}$          \\ \hline
        空心球,半径r,质量m                  & 过球心          &     $I=\dfrac{2mr^2}{3}$          \\ \hline
        \multirow{2}{*}{细杆,长l,质量m}   & 过杆中心,与杆垂直    &      $I=\dfrac{ml^2}{12}$         \\ \cline{2-3} 
                                     & 过杆顶点,与杆垂直    &     $I=\dfrac{ml^2}{3}$          \\ \hline
        \end{tabular}
    \end{table}

    计算刚体的转动惯量还有两个很重要的定理,\textbf{平行轴定理}和\textbf{垂直轴定理}。平行轴定理的描述如下,如果质量是m的刚体相对于一条过质心
    的旋转轴(简称为质心轴)的转动惯量是\(I_C\),如果有另一条平行于质心轴的旋转轴,且这条旋转轴距离质心轴的距离是d,那么刚体相对于这个旋转轴的转动惯量是
    \begin{equation*}
        I = I_C + md^2
    \end{equation*}
    
    垂直轴定理的描述如下,如果有一个x-y平面内的刚体(即刚体的厚度很小,可以忽略不计),假设刚体绕x轴和绕y轴的转动惯量分别是\(I_x\)和\(I_y\),
    那么其绕着z轴的转动惯量满足
    \begin{equation*}
        I_z = I_x + I_y
    \end{equation*}

    \begin{wraptable}{r}{6cm} 
        \centering
        \begin{tabular}{|c|c|}
            \hline
            \textbf{刚体转动} & \textbf{质点平动} \\ \hline
            角度$ \theta$            & 位移$ x$            \\ \hline
            角速度 $ \omega$          & 速度$ v$            \\ \hline
            转动惯量$ I$          & 质量$ m$            \\ \hline
            角动量$ L$           & 动量$ p$            \\ \hline
            力矩$ N$            & 力$ F$            \\ \hline
        \end{tabular}
    \end{wraptable}
    在本小节的最后,为了帮助大家更好地理解刚体中新出现的相关概念,我们可以将他们与质点中的物理量进行类比,需要注意的是,以下的类比不仅是物理量之间的类比,
    在对应的物理量之间也有相似的物理定律,例如角动量定理和动量定理。

    在这里,我们通过这种类比的方式,不加证明地给出刚体定轴转动的能量。假设刚体绕转轴的转动惯量是I,转动角速度是\(\omega\),则刚体转动动能是
    \begin{equation*}
        E = \frac{1}{2} I \omega^2
    \end{equation*}

    \textbf{例4.1.1}:有一个木棍,一端与固定转轴连接,木棍可以绕该转轴在某一水平面上自由转动。现在有一个子弹沿该水平面运动,并击中且嵌入了该木棒,随之在水平面中
    一同旋转,那么子弹集中木棍的哪个位置,使得在击中的一瞬间,转轴对木棍没有水平方向的力。

    这道题看似复杂,实际上就是同时运用了动量守恒和角动量守恒,通过将二者的方程联立来分析问题。假设木棍长l,质量M,子弹质量m,速度v,击中位置距离转轴长度为d。
    设击中后二者共同转动的角速度是\(\omega\),那么根据角动量守恒
    \begin{equation*}
        mvd = \frac{1}{3}Ml^2 \omega + m\omega d^2
    \end{equation*}
    假设转轴位置对木棍的力为0,则对整体有动量守恒
    \begin{equation*}
        mv = m\omega d + \frac{1}{2}Ml \omega
    \end{equation*}
    联立两式可得\(d=\dfrac{2}{3}l\)。这个结论在棒球运动中十分有用,击球手可以用球棒的大约2/3处击打棒球,使得自己手腕处受到的力最小。
    (当然,棒球棒不同位置粗细不同,所以应当有所调整)

    \subsection{质心和质心参考系}
    质心是一个大家既熟悉又陌生的概念,我们往往会把它和重心混淆,实际上,质心可以视作是质量m对于位矢的加权平均值,而重心是重力mg对于位矢的加权平均。
    当然,在研究的物体大小有限时,重力加速度g对于物体的每一处都是均匀的,因此质心和重心是重合的。接下来我们用比较严谨的数学语言去描述一个物体的质心。

    假设在一维的坐标上有n个质点,第i个质点的质量是\(m_i\),位矢(即坐标值)是\(x_i\),那么这n个质点的质心坐标是
    \begin{equation*}
        x_c = \frac{\displaystyle \sum_i^n m_i x_i}{\displaystyle \sum_i^n m_i}
    \end{equation*}
    如果是质量连续分布的物体,则需要改为积分形式
    \begin{equation*}
        x_c = \frac{\int x \, dm}{\int  dm}
    \end{equation*}

    这里,我们可以很自然地将质心位置的计算公式推广到三维空间中,对应的离散和连续质量分布的公式为
    \begin{align*}
        x_c = \frac{\displaystyle \sum_i^n m_i x_i}{\displaystyle \sum_i^n m_i}\ \ \ \ \ y_c =& \frac{\displaystyle \sum_i^n m_i y_i}{\displaystyle \sum_i^n m_i}\ \ \ \ \ z_c = \frac{\displaystyle \sum_i^n m_i z_i}{\displaystyle \sum_i^n m_i} \\
        x_c = \frac{\int x \, dm}{\int  dm}\ \ \ \ \ y_c =& \frac{\int y \, dm}{\int  dm}\ \ \ \ \ z_c = \frac{\int z \, dm}{\int  dm}
    \end{align*}

    对于三维空间里的积分,例如求\(x_c\)的值时,可以视为是将\(x=x_0\)平面内的所有质量都汇聚在了\((x_0,0,0)\)这一点,
    然后再按照一维积分的方式去处理。

    \textbf{例4.2.1}:求一个半径是r的均匀半圆盘的质心位置。

    我们将半圆盘放在坐标系中,使得圆心和坐标中心重合,半圆直边和x轴重合。由对称性可得,质心一定在y轴上,我们只需要求其y坐标值。
    这里半圆盘均匀,我们用面积来表示质量,则高度为y处的微元dy所对应的质量是\(dm = 2\sqrt{r^2 - y^2}dy\),带入质心求解公式
    \begin{align*}
        y_c &= \frac{\int 2y\sqrt{r^2 - y^2}dy }{\frac{1}{2}\pi r^2 }
    \end{align*}
    求解可得\(y_c = \dfrac{4r}{3\pi}\),故半圆盘的质心在\((0,\dfrac{4r}{3\pi})\)处。

    确定了质心的位置,接下来我们研究一下质心的运动。对于质心位置的求解公式,我们在方程两边分别对时间求一阶和二阶导数,得到速度和加速度的关系。有
    \begin{equation*}
        v_c = \frac{\displaystyle \sum_i^n m_i v_i}{\displaystyle \sum_i^n m_i}\ \ \ \ \ \ \ \ a_c = \frac{\displaystyle \sum_i^n m_i a_i}{\displaystyle \sum_i^n m_i}
    \end{equation*}
    回忆一下质点系的牛顿第二定律的内容,我们注意到质心的加速度满足
    \begin{equation*}
        \sum_i^n m_i a_c = \sum_i^n F_i
    \end{equation*}
    其中\(F_i\)是第i个质点所受的和外力。于是,质心的牛顿第二定律可描述为,质心的加速度等于刚体受到的总外力除以总质量。

    刚体的物理性质决定了,刚体每一处的平动速度和平动加速度都相等,自然也就和质心的速度和加速度相等。可见,质心的运动对于刚体来说有很大的特殊性,
    于是我们可以选择质心作为参考系,即\textbf{质心参考系},在质心参考系中分析刚体的运动。

    我们假设刚体的每一处质点都有相同的平动速度和平动加速度(例如刚体在重力场中自由落体),那么显然质心系是一个非惯性系。在这个非惯性系中,每一个质点受到的\textbf{惯性力产生的加速度}都正好与其
    \textbf{原有的平动加速度}抵消,当然,每一个质点相对于质心的平动速度也是0。于是我们得到一个很重要的结论——在质心系中刚体只有旋转运动。

    利用这个结论,我们考虑一般情况下刚体的动能计算,刚体的运动可以视为质心的平动和整体相对于质心的转动,于是刚体的动能就是质心的平动动能加质心系中的转动动能,这就是\textbf{柯尼希定理}。用数学表达式描述
    \begin{equation*}
        E_k = \frac{1}{2}mv_c^2 + \frac{1}{2} I_c \omega^2
    \end{equation*}
    
    \subsection{物体的平衡}
    刚体的平衡需要满足以下两个条件,刚体所受的和外力为0,以及刚体的和力矩为0。用数学语言可描述为

    \[
        \begin{cases}
            \displaystyle \sum_i^n \vec{F_i} = 0 \\
            \displaystyle \sum_i^n \vec{N_i} = 0 
        \end{cases}
    \]
    
    这里我们自然要问一个问题,上式的力矩平衡,是相对于哪个转轴的。答案是对任何一个转轴都要满足力矩平衡,实际上,
    如果刚体的和外力为0,且存在一个转轴使得和力矩为0,那么可以证明其对于任何一个转轴的和力矩都是0。我们假设有n个外力,
    第i个外力的力臂是\(r_i\),故有
    \begin{align*}
        \sum_i^n &\vec{F_i} = 0 \\
        \sum_i^n \vec{r_i} &\times \vec{F_i} = 0
    \end{align*}
    假设旋转轴位置改变,使得第i个力的力臂变为\(\vec{r_i} + \vec{r^\prime}\),那么新的力矩之和为
    \begin{align*}
          & \sum_i^n (\vec{r_i} + \vec{r^\prime}) \times \vec{F_i} \\
        = & \sum_i^n \vec{r_i} \times \vec{F_i} + \vec{r^\prime} \times \sum_i^n \vec{F_i}
    \end{align*}

    由于\(\displaystyle \sum_i^n \vec{F_i} = 0\)和\(\displaystyle \sum_i^n \vec{r_i} \times \vec{F_i} = 0\),故上式等于0,得到关于所有的旋转轴力矩都是0。

    在解决平衡问题时,我们需要巧妙选择旋转轴的位置,使得尽可能多的未知力的力矩为0,从而大大化简我们的分析和计算。

    下面介绍物体平衡的种类,物体的平衡分为三类,稳定平衡,不稳定平衡和随遇平衡。
    \footnote{随遇平衡是一个更通俗的说法,专业叫法是中性平衡。不过我更喜欢随遇平衡这个名字,这让人想到“随遇而安”}
    三种平衡的区别是,在物体受到微小扰动后,是否能够保持平衡。稳定平衡是物体在微扰后可以在原处保持平衡,例如放在碗底的小球。不稳定平衡的物体在受到扰动后其平衡会被破坏,
    例如稳定在圆锥顶部的小球。而随遇平衡则是物体在受到微扰后,会在新的位置重新保持平衡,例如放在水平地面上的小球。

    “微扰”的说法总是让人很困惑,在具体问题中,想要利用“微扰”来分析问题也很困难,这里我们换一种考虑问题的方式。假设物体的平衡状态是由保守力决定的(很多时候只需要忽略摩擦力就行),
    那么物体在每一个位置都会对应一个势,这个势是所有保守力对应的势的叠加。在一个一维空间中,如果我们知道了势对于位置的函数,那么对其求导就得到了力对于位置的函数。
    平衡位置也就是势函数的导数等于0,而势函数的极小值点对应稳定平衡,极大质点对应不稳定平衡,在函数值为常数的区间上对应随遇平衡。这样,我们就将求平衡种类的问题变成了求势能的问题。

    


\end{document}
